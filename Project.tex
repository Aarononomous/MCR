\documentclass{amsart}
\usepackage{graphicx}
\usepackage{textcomp}

\begin{document}

{\Large

\title{Minimum Cover Removal Project}
\author{Aaron Jacobson}
\date{March 2017}
\maketitle

\begin{abstract}
[[ Abstract goes here ]]
\end{abstract}

\section{Introduction}

Parametrization by ...

\section{Analysis}

\section{Ideas}

\subsection{Analysis}

How does the planarity of the graph affect graph traversal? How can I use this to my advantage?

\subsection{Approximation Algorithms}

One way to approach this is with path traversal. An opposite way is with obstacle removal. Is there a way to blend these? What's a way of measuring the ``goodness'' of obstacle removal?

\textbf{The shapes closest to the start/goal:} The algorithms of Hauser walk through the graph from $s$. A more-interesting approach would be an incremental approach that \emph{doesn't} use BFS. However, this is a good and obvious first approach.

\textbf{Remove/special case for obstacles with only pairwise intersections:} I don't have any proof for this yet.

\textbf{What's the quickest way to prune?}

\textbf{Dynamic Programming Approaches:} [[[ TODO ]]]

Which shape, when removed, reduces the 
\emph{largest} number of contiguous \emph{free} spaces; i.e., removing it \emph{connects} free spaces? I feel like this won't work well---how to prove that.

\textbf{Ordering obstacles by number of overlaps:} In a graph, this is the number of multi-labeled vertices.

\textbf{Ordering obstacles by total number of vertices:} See above. The difference is unlabeled adjacencies. Would this be better? What about ordering by number of unlabeled adjacencies, then?

\textbf{bounded by the number of clear spaces that are passed through in a minimal path???} I don't feel like this would work. [[[ TODO: Analyze this. ]]]

For any two vertices in the planar graph, what is the minimum number of obstacles needed to be removed to connect them?

\subsection{Parameterizations}

\textbf{Total number of overlaps:}

\textbf{Average number of overlaps:}

\textbf{Initial number of paths:} This is really interesting. How do I measure this?

\textbf{Average path cost:} Again, how do I measure this? I guess since I have to iterate through all paths anyways, I can just count.

\textbf{Number of unique MCR's}

\textbf{Each node is valued by the total cover of its neighborhood:} $f: \mathbb{N} = v \to |\bigcup_{u \in Adj(v)} C(u) |$

\textbf{Repeat Entry Penalization:} If a subsequent entry into obstacle $O$ is penalized with 1, then the problem is in P, since we'll just be finding a path through the fewest obstacles. (Q: How is this affected by re-entry within overlaps?) What about slowly lowering the penalty by $\varepsilon$ for repeat entries? What's a good amount to reduce by? This boils down to path augmentation in the middle, which is interesting.

\textbf{Shortest Path Approx.:} For every obstacle vertex, add an edge of weight *intersections*. What's the shortest path through this?

\subsection{Ideas That Don't Work}

\textbf{Prioritizing Largest Areas:} There's no cost except in entering obstacles for the first time; this is why we can reduce MCR to a discrete problem, and ultimately, to a problem in a graph. Varying an obstacle's size will only affect the runtime if it varies the number of overlaps covered by the obstacle.

\textbf{Can we mark some obstacles as \emph{necessary} to be removed?:} Sure. These are shapes that either a) overlap $s$; b) overlap $t$; or c) overlay the entire field in the x- or y-direction. All three of these conditions can be tested in $O(n) O(1)$. This might remove some low-hanging fruit, but it doesn't affect the overall tractability of the problem.

\textbf{Pairwise Intersection:} "The problem becomes trivial when the obstacles are allowed to intersect only pairwise" [[ Cite ]]

Proof: A conversion from pairwise intersection in the plane to a partition in an induced graph $G$ creates two sorts of labeled vertices: those with one label and those with two. The vertices with two labels $a, b$ can only be connected to vertices labeled $a$ and $b$. Where the obstacles do not intersect at their sides, the induced vertex is the only vertex on the path $(a, b)$, and since proceeding through it can only be done while on a path through both the $a$ and $b$ vertices, we can erase the redundant label and merge edges $(a, (a, b))$ and $((a, b), b)$. On a side-intersection, the MCR path from start to goal will only need to pass through both obstacles if it begins in one; since that will have to be removed anyways, the degenerate case is obviated. After merging edges, we can continue to contract the graph to reach a minor with all vertices labeled---this is now reduced to finding a shortest path in $G$.


[[[ Can this be used to remove any obstacles? ]]]

\section{Todo}

- import SVG's
X label polygons with colors, add'l info?
X label polygons within the display, using obj.representativepoint

Shape intersection to new shapes
    - how will this work exactly?
    Idea 1. Overlay the shapes one by one on the artboard: add each shape to the graph as it's entered; problem: this will cetainly create convex shapes that I need to check intersection on.
    Idea 2. Add all shapes as outlines; create new segments from each intersection; add all created shapes to the graph then label them by the original shapes
    - check for a good algorithm for shape intersection
    - shapely prepared.prep
    
graph library

add tolerance for point comparisons

shape adjacency to (labeled) graph - partition of the plane
    - special case where $s$ and $g$ are in the same partition

SVG to shapes
a. resize to [1, 1]
√. display as translucent
b.1. When ordering obstacles by property x in some algorithm, can we color them to show the order?

overlay graph onto shapes

develop the optimal algorithm from Hauser

develop the greedy algorithm from Hauser

develop the same random shapes and tests as in Hauser

graph results

display an example of the worst-case $|G| 2^{|S^*|}$ states configuration - for the optimal algorithm

display a worst-case example for the greedy algorithm -- Hauser won't work; his example is a concave shape

find LaValle 2006

find Choset et al. 2005


\subsection{For Later}

convert MAX2HORNSAT into an MCR

Am I going to run into any floating-point troubles?

Do I need to worry a lot about C.G. problems like shapes intersecting at a line/point?

What about taking concave shapes and splitting them into convex -- is it useful to illustrate examples or a waste of time?

I'll need one to some notebooks to introduce the problem and the machinery. How do I import these into further notebooks?
  
  http://nbviewer.jupyter.org/gist/minrk/6011986

\section{Approach}

\subsection{Mathematical Stuff}

- overlapping shapes to a labelled graph

When the start/goal are within a shape/shapes, would it be best to have a single labeled vertex which connects to the overlap area? Does this affect the order of operations?

Using Jupyter


\section{Appendix}

[[ Code here? ]]

\end{document}


